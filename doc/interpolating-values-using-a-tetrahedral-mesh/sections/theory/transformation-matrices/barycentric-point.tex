\subsubsection{Barycentric Coordinates of a Point Projection}

In the case of three vertices of  the  tetrahedron  being located infinitely far
away, any 3D point located  inside its volume will effectively be projected onto
the one finite vertex.

The  3  dimensional Cartesian coordinate $\vec{x}\in\mathbb{R}^3$  is  projected
onto one of the tetrahedron's vertices $v_1 \in \mathbb{R}^3$. This means one of
the  barycentric  coordinates will always be 1 and the other  three  barycentric
coordinates will have value 0.  This  can  be  described with the transformation
matrix:

\begin{equation}
    \label{eq:point}
    P = \begin{bmatrix}
        0 & 0 & 0 & \lambda_1 \\
        0 & 0 & 0 & \lambda_2 \\
        0 & 0 & 0 & \lambda_3 \\
        0 & 0 & 0 & \lambda_4 \\
    \end{bmatrix}
\end{equation}

Where one of the entries in $\vec{\lambda}$ has  value  1  and all other entries
have value 0.

It should be noted that a projection  matrix may be overkill for this particular
case. However, it is necessary if one wants to generalise  the implementation by
using a 4x4 matrix.

